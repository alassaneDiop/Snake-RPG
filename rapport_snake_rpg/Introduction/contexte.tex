\par
Dans le cadre de l'UE Design Pattern, de notre premier semestre de Master 1 Informatique à l'Université d'Angers, nous avons poursuivi la phase d'analyse d'un développement de jeu de Snake en réseau.

\par
Le jeu peut se jouer en local ou en réseau, entre différents joueurs ce qui nécessite une implémentation des protocoles réseaux, pour gérer les transmissions entre les joueurs mais aussi entre les joueurs et le serveur.\\
Ceci nous a amené à mettre en place une sorte de document RFC\footnote{Les requests for comments (RFC), littéralement « demande de commentaires », sont une série numérotée de documents officiels décrivant les aspects techniques d'Internet, ou de différents matériels informatiques (routeurs, serveur DHCP).Wikipédia} pour expliquer les protocoles réseaux utilisés et comment les utiliser.

\par
La conception de ce jeu est assez complexe, raison pour laquelle il nous a fallu réfléchir sur le choix des technologies et pour cela nous avons préféré adopter un patron de conception\footnote{Un patron de conception (souvent appelé design pattern) est un arrangement caractéristique de modules, reconnu comme bonne pratique en réponse à un problème de conception d'un logiciel. Il décrit une solution standard, utilisable dans la conception de différents logiciels. Wikipédia} pour garantir une solution éprouvée et validée par des experts, mais aussi trouver une solution avec les bonnes pratiques de conception.