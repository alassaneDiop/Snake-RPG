\par
Les échanges entre le serveur et les joueurs reposent sur la norme ISO-8859\footnote{ISO 8859, également appelée plus formellement ISO/CEI 8859, est une norme commune de l'ISO et de la CEI de codage de caractères sur 8 bits pour le traitement informatique du texte. Wikipédia} qui, a l'avantage d'être connue par tous les systèmes d'exploitation et ne requiert que 8 bits par caractère. \\
8 bits = 1 byte \\

\par
Les commandes des joueurs commenceront par le caractère '/' suivi du nom de la commande. \\
Exemple : \textit{/quit} \\
Cette commande permet à un joueur de quitter une partie de jeu. \\

\par
Toutes les réponses du serveur seront représentées comme suit : 

\begin{itemize}
	\item \textit{1xx} réponse de succès d'une commande
	\item \textit{2xx} réponse d'erreur d'une commande 
	\item \textit{3xx} Commande du serveur \\
\end{itemize}

\newpage
\par
Pour simplifier le transfert de données et éviter des conflits avec le XML (voir \ref{xml}), les caractères suivants ne seront pas acceptés : 

\begin{itemize}
	\item '"' guillemets
	\item '<' signe d'infériorité
	\item '>' signe de supériorité \\
\end{itemize}

Le caractère \textit{' '} (espace) est autorisé, s'il s'agit de séparer des blocs de données. Une commande ou un paramètre ne peut contenir de caractère \textit{' '} (espace). \\

Toutes les commandes devront terminer par le caractère \textbackslash n pour marquer la fin de la commande. \\