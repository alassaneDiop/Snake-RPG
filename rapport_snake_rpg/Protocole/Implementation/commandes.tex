
%All the responses of the server are beginning with a numeric character 
%as follow:
 %   – 1xx Accept command response
  %  – 2xx Denied command response
   % – 3xx Server command



\par
Les échanges entre le serveur et les joueurs reposent sur la norme ISO-8859\footnote{ISO 8859, également appelée plus formellement ISO/CEI 8859, est une norme commune de l'ISO et de la CEI de codage de caractères sur 8 bits pour le traitement informatique du texte. Wikipédia}. \\

8 bits = 1 byte \\

\par
Les commandes des joueurs commenceront par le caractère '/' suivi du nom de la commande. \\
Exemple : $/quit$ \\
Cette commande permet à un joueur de quitter une partie de jeu. \\

Pour simplifier le transfert de données et éviter des conflits avec le XML (voir \ref{xml}), les caractères suivants ne seront pas acceptés : 

\begin{itemize}
	\item '"' guillemets
	\item '<' signe d'infériorité
	\item '>' signe de supériorité \\
\end{itemize}

Le caractère '$~$' (espace) est autorisé s'il s'agit de séparer des blocs de données. Une commande ou un paramètre ne peuvent contenir un caractère '$~$' (espace). \\

Toutes les commandes devront terminer par le caractère \textbackslash n pour marquer la fin de la commande.